\documentclass{article} % A4 paper and 11pt font size
\setcounter{secnumdepth}{0}

\usepackage{amssymb, amsmath, amsfonts}
\usepackage{moreverb}
\usepackage{graphicx}
\usepackage{enumerate}
\usepackage{graphics}
\usepackage[margin=1in]{geometry}
\usepackage{color}
\usepackage{tocloft}
\renewcommand{\cftsecleader}{\cftdotfill{\cftdotsep}}
\usepackage{array}
\usepackage{float}
\usepackage{csquotes}
\usepackage{verbatim}
\usepackage{hyperref}
\usepackage{textcomp}
\usepackage[makeroom]{cancel}
\usepackage{bbold}
\usepackage{scrextend}
\usepackage{alltt}
\usepackage{listings}
\usepackage{physics}
\usepackage{mathtools}
\usepackage[normalem]{ulem}
\usepackage{amsthm}
\usepackage{tikz}
\usetikzlibrary{positioning}
\usetikzlibrary{arrows}
\usepackage{pgfplots}
\usepackage{bigints}
\allowdisplaybreaks
\pgfplotsset{compat=1.12}

\theoremstyle{plain}
\newtheorem*{theorem*}{Theorem}
\newtheorem{theorem}{Theorem}
\newtheorem*{lemma*}{Lemma}
\newtheorem{lemma}{Lemma}

\definecolor{verbgray}{gray}{0.9}
% \definecolor{dkgreen}{green}{0.9}

\lstnewenvironment{code}{%
  \lstset{
  language=R,
  backgroundcolor=\color{verbgray},
  keywordstyle=\color{blue},      % keyword style
  commentstyle=\color{magenta},   % comment style
  stringstyle=\color{olive},      % string literal styleframe=single,
  numberstyle=\color{black},      % string literal styleframe=single,
  framerule=0pt,
  numbers=left,
  stepnumber=1,
  firstnumber=1,
  showspaces=false,
  basicstyle=\ttfamily}}{}

\lstnewenvironment{console_output}{%
  \lstset{
  framerule=0pt,
  numbers=left,
  stepnumber=1,
  showspaces=false,
  firstnumber=1,
  basicstyle=\ttfamily}}{}


\makeatletter
\newcommand{\BIGG}{\bBigg@{3}}
\newcommand{\vast}{\bBigg@{4}}
\newcommand{\Vast}{\bBigg@{5}}
\makeatother

\newenvironment{definition}[1][Definition]{\begin{trivlist}
\item[\hskip \labelsep {\bfseries #1}]}{\end{trivlist}}

\newcommand{\dy}{\partial_y}
\newcommand{\dyy}{\partial_{yy}}
\newcommand{\dxx}{\partial_{xx}}
\newcommand{\dxy}{\partial_{xy}}
\newcommand{\dyyy}{\partial_{yyy}}
\newcommand{\dxxx}{\partial_{xxx}}
\newcommand{\dx}{\partial_x}
\newcommand{\E}{\varepsilon}
\def\Rl{\mathbb{R}}
\def\Cx{\mathbb{C}}

\newcommand{\Ei}{\text{Ei}}

\usepackage[T1]{fontenc} % Use 8-bit encoding that has 256 glyphs
\usepackage{fourier} % Use the Adobe Utopia font for the document - comment this line to return to the LaTeX default
\usepackage[english]{babel} % English language/hyphenation

\usepackage{sectsty} % Allows customizing section commands
\allsectionsfont{\centering \normalfont\scshape} % Make all sections centered, the default font and small caps

\usepackage{fancyhdr} % Custom headers and footers
\pagestyle{fancy} % Makes all pages in the document conform to the custom headers and footers
\fancyhead[L]{\bf Sam Fleischer}
\fancyhead[C]{\bf UC Davis \\ Numerical Solutions of Differential Equations (MAT228A)} % No page header - if you want one, create it in the same way as the footers below
\fancyhead[R]{\bf Fall 2016}

\fancyfoot[L]{\bf } % Empty left footer
\fancyfoot[C]{\bf \thepage} % Empty center footer
\fancyfoot[R]{\bf } % Page numbering for right footer
\renewcommand{\headrulewidth}{0pt} % Remove header underlines
\renewcommand{\footrulewidth}{0pt} % Remove footer underlines
\setlength{\headheight}{25pt} % Customize the height of the header

\newcommand{\VEC}[2]{\left\langle #1, #2 \right\rangle}
\newcommand{\ran}{\text{\rm ran }}
\newcommand{\Hilb}{\mathcal{H}}
\newcommand{\lap}{\Delta}

\newcommand{\littleo}[1]{\text{\scriptsize$\mathcal{O}$}\qty(#1)}

\DeclareMathOperator*{\esssup}{\text{ess~sup}}

\newcommand{\problem}[2]{
\vspace{.375cm}
\boxed{\begin{minipage}{\textwidth}
    \section{\bf #1}
    #2
\end{minipage}}
}

\numberwithin{equation}{section} % Number equations within sections (i.e. 1.1, 1.2, 2.1, 2.2 instead of 1, 2, 3, 4)
\numberwithin{figure}{section} % Number figures within sections (i.e. 1.1, 1.2, 2.1, 2.2 instead of 1, 2, 3, 4)
\numberwithin{table}{section} % Number tables within sections (i.e. 1.1, 1.2, 2.1, 2.2 instead of 1, 2, 3, 4)

\setlength\parindent{0pt} % Removes all indentation from paragraphs - comment this line for an assignment with lots of text

\newcommand{\horrule}[1]{\rule{\linewidth}{#1}} % Create horizontal rule command with 1 argument of height

\title{ 
\normalfont \normalsize 
\textsc{UC Davis, Numerical Solutions of Differential Equations (MAT 228A), Fall 2016} \\ [25pt] % Your university, school and/or department name(s)
\horrule{2pt} \\[0.4cm] % Thin top horizontal rule
\Huge Homework \#1 \\ % The assignment title
\horrule{2pt} \\[0.5cm] % Thick bottom horizontal rule
}

\author{\huge Sam Fleischer} % Your name

\date{October 14, 2016} % Today's date or a custom date

\begin{document}\thispagestyle{empty}

\maketitle % Print the title

\makeatletter
\@starttoc{toc}
\makeatother

\pagebreak

%%%%%%%%%%%%%%%%%%%%%%%%%%%%%%%%%%%%%%
\problem{Problem 1}{Let $L$ be the linear operator $Lu = u_{xx}$, $u_x(0) = u_x(1) = 0$.
\begin{enumerate}[\ \ (a)]
    \item Find the eigenfunctions and corresponding eigenvalues of $L$.
    \item Show that the eigenfunctions are orthogonal in the $L^2[0,1]$ inner product $$\langle u, v \rangle = \int_0^1 u(x) v(x) \dd x.$$
    \item It can be shown that the eigenfunctions $\phi_j(x)$, form a complete set in $L^2[0,1]$. This means that for any $f \in L^2[0,1]$, $f(x) = \sum_{j} \alpha_j\phi_j(x)$.  Express the solution to $$u_{xx} = f, u_x(0)=u_x(1) = 0,$$ as a series solution of the eigenfunctions.
    \item Note that this BVP does not have a solution for all $f$.  Express the condition for existence of a solution in terms of the eigenfunctions of $L$.
\end{enumerate}}

\begin{enumerate}[\ \ (a)]
    \item
        Let $Lu = \lambda u$.  Then
        \begin{align*}
            u_{xx} - \lambda u = 0 \\
            \implies u(x) = A\exp[\sqrt{\lambda}x] + B\exp[-\sqrt{\lambda}x]
        \end{align*}
        If $\lambda > 0$ then $u_x(x) = \sqrt{\lambda}\qty[A\exp[\sqrt{\lambda}x] - B\exp[\sqrt{-\lambda}x]]$ and the boundary condition $u_x(0) = 0$ implies $A = B$ and so $u_x(x) = A\sqrt{\lambda}\qty[\exp[\sqrt{\lambda}x] - \exp[-\sqrt{\lambda}x]]$.  Then the boundary condition $u_x(1) = 0$ implies $A = 0$, thus there are no solutions.

        If $\lambda = 0$ then $u(x) = A + Bx$ and the boundary conditions implies $B = 0$.  Thus $u(x) = A$ where $A$ is a constant is an eigenfunction.

        If $\lambda < 0$ then $u(x) = A\sin(\sqrt{-\lambda}x) + B\cos(\sqrt{-\lambda}x)$ and thus $u_x(x) = \sqrt{-\lambda}\qty[A\cos(\sqrt{-\lambda}x) - B\sin(\sqrt{-\lambda}x)]$.  Then the boundary conditions imply $A = 0$ and $\sqrt{-\lambda} = k\pi$ for $k \in \mathbb{N}$.  Thus the eigenfunctions are
        \begin{align*}
            u_k(x) = \cos(k\pi x) \qquad \text{for $k = 0, 1, 2, \dots$ with corresponding eigenvalues } \lambda_k = -k^2\pi^2
        \end{align*}
    \item
        Assume $k \neq j$.  Then
        \begin{align*}
            \left\langle \cos(k \pi x), \cos(j \pi x) \right\rangle &= \int_0^1\cos(k\pi x)\cos(j\pi x)\dd x \\
            &= \frac{1}{2}\int_0^1\cos((k+j)\pi x) + \cos((k-j)\pi x)\dd x \\
            &= \frac{1}{2}\qty[\left.\frac{\sin(k+j)\pi x}{(k+j)\pi}\right|_0^1 + \left.\frac{\sin(k-j)\pi x}{(k-j)\pi}\right|_0^1] \\
            &= \frac{1}{2(k+j)\pi}\qty[\sin(k+j)\pi] + \frac{1}{2(k-j)\pi}\qty[\sin(k-j)\pi]
        \end{align*}
        Since $k \neq j$ then $k+j$ and $k-j$ are nonzero integers and thus $\left\langle u_k, u_j\right\rangle = 0$.  If $k = j$, then
        \begin{align*}
            \left\langle \cos(k\pi x), \cos(k\pi x)\right\rangle = \int_0^1\cos^2\qty(k\pi x)\dd x = \frac{1}{2}\int_0^1 1 + \cos(2k\pi x)\dd x = \frac{1}{2}
        \end{align*}
        Thus $\sqrt{2}u_k$ are orthonormal.
    \item
        Let $u = \displaystyle\sum_j \alpha_j\cos(j\pi x)$, and let $f = \displaystyle\sum_j \beta_j\cos(j\pi x)$.  Then $u_{xx} = -\displaystyle\sum_j \alpha_jj^2\pi^2\cos(j\pi x)$.  Then
        \begin{align*}
            -\sum_j\alpha_j j^2\pi^2\cos(j\pi x) = \sum_j \beta_j\cos(j\pi x)
        \end{align*}
        Since $\cos(j\pi x)$ are orthonormal, each term in the series must match, and thus
        \begin{align*}
            \alpha_j = -\frac{\beta_j}{j^2\pi^2}
        \end{align*}
        So, the solution to $Lu = f$ where $f = \displaystyle\sum_j \beta_j \cos(j\pi x)$ is
        \begin{align*}
            u(x) = -\sum_j \frac{\beta_j}{j^2\pi^2}\cos(j\pi x)
        \end{align*}
    \item
        Since $L$ is self-adjoint, then $Lu = f$ is solvable if $f \perp \ker L$ where $\ker L = \qty[1]$, that is, the kernel of $L$ is spanned by the constant function $1$.  $f \perp \ker L$ if
        \begin{align*}
            \left\langle f, 1\right\rangle = \int_0^1 f(x)\dd x = 0
        \end{align*}
        that is, the mean of $f(x)$ is zero.  In terms of the eigenfunctions,
        \begin{align*}
            \left\langle f, \cos(0\pi x)\right\rangle = 0.
        \end{align*}
\end{enumerate}

\problem{Problem 2}{Define the functional $F\ :\ X \rightarrow \Rl$ by $$F(u) = \int_0^1 \frac{1}{2}(u_x)^2 + fu \dd x,$$ where $X$ is the space of real-valued functions on $[0,1]$ that have at least one continuous derivative and are zero at $x = 0$ and $x = 1$.  The Frechet derivative of $F$ at a point $u$ is defined to be the linear operator $F'(u)$ for which $$F(u+v) = F(u) + F'(u)v + R(v),$$ where $$\lim_{\norm{v}\rightarrow 0}\frac{\norm{R(v)}}{\norm{v}} = 0.$$  One way to compute the derivative is $$F'(u)v = \lim_{\E \rightarrow 0}\frac{F(u + \E v) - F(u)}{\E}.$$ Note that this looks just like a directional derivative.
\begin{enumerate}[\ \ (a)]
    \item Compute the Frechet derivative of $F$.
    \item $u \in X$ is a critical point of $F$ if $F'(u)v = 0$ for all $v \in X$.  Show that if $u$ is a solution to the Poisson eqution $u_{xx} = f$, $u(0) = u(1) = 0$, then it is a critical point of $F$.
    \item Let $X_h$ be a finite dimensional subspace of $X$, and let $\{\phi_i(x)\}$ be a basis for $X_h$.  This means that all $u_h \in X_h$ can be expressed as $u_h(x) = \sum_iu_i\phi_i(x)$ for some constants $u_i$.  Thus we can identify the elements of $X_h$ with vectors $\vec{u}$ that have components $u_i$.  Let $G(\vec{u}) = F(u_h)$.  Show that the gradient of $G$ (whos components are $(\grad G)_j = \frac{\partial G}{\partial u_j})$ is of the form $\grad G(\vec{u}) = A\vec{u} + \vec{b}$, and write expressions for the elements of the matrix $A$ and the vector $\vec{b}$.
    \item Divide the unit interval into a set of $N+1$ equal length intervals $I_i = (x_i,x_{i+1})$ for $i = 0, \dots, N$.  The enpoints of the intervals are $x_i = ih$, where $h = \frac{1}{N+1}$.  Let $X_h$ be the subspace of $X$ such that the elements $u_h$ of $X_h$ are linear on each interval, continuous on $[0,1]$, and satisfy $u_h(0) = u_h(1) = 0$.  $X_h$ is an $N$ dimensional space with basis elements
    \begin{align*}
        \phi_i(x) = \begin{cases}
            1 - h^{-1}\abs{x - x_i} & \text{ if } \abs{x - x_i} < h,\\
            0 & \text{ otherwise}
        \end{cases}
    \end{align*}
    for $i = 1, \dots, N$.  Compute the matrix $A$ from the previous problem that appears in the gradient.
\end{enumerate}
Finite element methods are based on these ``weak formulations'' of the problem.  The Ritz method is based on minimizing $F$ and the Galerkin method is based on finding the critical points of $F'(u)$.}

\begin{enumerate}[\ \ (a)]
    \item
        \begin{align*}
            F'(u)v &= \lim_{\E \rightarrow 0}\frac{F(u + \E v) - F(u)}{\E} \\
            &= \lim_{\E \rightarrow 0}\frac{\int_0^1\frac{1}{2}\qty(u_x + \E v_x)^2 + f(u + \E v) - \frac{1}{2}u_x^2 - f u\dd x}{\E} \\
            &= \lim_{\E\rightarrow 0} \int_0^1 u_xv_x + \frac{1}{2}\E v_x^2 + f v \dd x \\
            &= \int_0^1 u_x v_x + f v \dd x \\
            &= \qty[v u_x]_0^1 + \int_0^1 v(f - u_{xx}) \dd x
        \end{align*}
        But $v \in X$, and thus $v(0) = v(1) = 0$.  So, for all $v$,
        \begin{align*}
            F'(u) v &= \int_0^1 v\qty(f - u_{xx})\dd x
        \end{align*}
    \item
        Let $u$ be a solution to the Poisson equation.  Then $u_{xx} = f$.  Then $u_{xx} - f = 0$.  Thus,
        \begin{align*}
             F'(u)v = \int_0^1 v(f - u_{xx}) \dd x = \int_0^1 v\cdot 0\dd x = 0
        \end{align*}
    \item
        \begin{align*}
            G(\vec{u}) = F(u_h) &= \int_0^1 \frac{1}{2}\qty(\sum_{i=1}^n u_i \phi_i'(x))^2 + f(x) \sum_{i=1}^n u_i\phi_i(x) \dd x \\
            \implies \qty(\grad G)_j = \frac{\partial}{\partial u_j} G(\vec{u}) &= \int_0^1 \qty(\sum_{i=1}^n u_i \phi_i(x))\phi'_j(x) + f(x) \phi_j(x)\d x \\
            &= \qty[\phi_j\sum_{i=1}^nu_i\phi_i']_0^1 + \int_0^1\phi_j(x)\qty(f(x) - \sum_{i=1}^nu_i\phi''_i(x))\dd x
        \end{align*}
        But the boundary conditions are $0$ since $\phi_j \in X_h \subset X$ and $u(0) = u(1) = 0$ for all $u \in X$.  Thus,
        \begin{align*}
            \qty(\grad G)_j &= - \int_0^1\qty(\sum_{i=1}^n u_i\phi''_i(x))\phi_j(x)\dd x + \int_0^1 f(x) \phi_j(x) \dd x \\
            &= A\vec{u} + \vec{b}
        \end{align*}
        where $\vec{b} = \qty(\begin{array}{c}
            \int_0^1 f(x) \phi_1(x) \dd x \\
            \int_0^1 f(x) \phi_2(x) \dd x \\
            \vdots \\
            \int_0^1 f(x) \phi_n(x) \dd x
        \end{array})$ and $A = (a_{ij})$ where $\displaystyle a_{ij} = -\int_0^1 \phi_i(x)\phi''_j(x) \dd x$.
    \item
        Note that
        \begin{align*}
            \phi_i(x) = \begin{cases}
                1 - h^{-1}\abs{x - x_i} & \text{ if } \abs{x - x_i} < h,\\
                0 & \text{ otherwise}
            \end{cases}
        \end{align*}
        and thus
        \begin{align*}
            \phi'_i(x) = \begin{cases}
                \frac{1}{h} & \text{ if } x_{i-1} < x < x_i \\
                -\frac{1}{h} & \text{ if } x_i < x < x_{i+1} \\
                0 & \text{ otherwise}
            \end{cases} \qquad \text{and} \qquad \phi''_i(x) = \delta\qty(x - \frac{i}{N+1})
        \end{align*}
        This shows
        \begin{align*}
            a_{ij} &= -\int_0^1 \phi_i(x)\delta\qty(x - \frac{j}{N+1})\dd x = -\phi_i\qty(\frac{j}{N+1}) = \begin{cases}
                -1 & \text{ if } i = j \\
                0 & \text{ otherwise}
            \end{cases}
        \end{align*}
        and so $A = -I$ where $I$ is the $n\times n$ identity matrix.
\end{enumerate}

\problem{Problem 3}{\begin{enumerate}[\ \ (a)]
    \item Using a Taylor expansion, derive the finite difference formula to approximate the second derivative at $x$ using function values at $x - \frac{h}{2}$, $x$, and $x + h$.  How accurate is the finite difference approximation?
    \item Perform a refinement study to verify the accuract of the difference formula you derived.
    \item Derive an expression for the quadratic polynomial that interpolates the data $\qty(x - \frac{h}{2}, u\qty(x - \frac{h}{2}))$, $(x, u(x))$, and $(x + h, u(x + h))$.  How is the finite difference formula you derived in problem 3a related to the interpolating polynomial?
\end{enumerate}}

\begin{enumerate}[\ \ (a)]
    \item
        Let $D^2$ be a finite difference formula for the secon derivative.  Then
        \begin{align}
            (D^2u)_j &= au\qty(x - \frac{h}{2}) + bu\qty(x) + cu\qty(x + h) \\
            &= (a + b + c)u(x) + \qty(c - \frac{a}{2})hu'(x) + \qty(\frac{a}{8} + \frac{c}{2})h^2 u''(x) + \qty(\frac{c}{6} - \frac{a}{48})h^3u'''(x) + \dots
        \end{align}
        The letting $a + b + c = 0$, $c = \frac{a}{2}$ and $a = \frac{8}{3h^2}$, we get
        \begin{align}
            (D^2u)_j &= u''(x) + \frac{h}{6}u'''(x) + \dots
        \end{align}
        where $a = \frac{8}{3h^2}$, $b = -\frac{4}{h^2}$, and $c = \frac{4}{3h^2}$.  Thus
        \begin{align}
            (D^2 u)_j = \frac{8}{3h^2}u\qty(x - \frac{h}{2}) - \frac{4}{h^2}u(x) + \frac{4}{3h^2}u(x + h)
        \end{align}
        {\color{red} not done...}
    \item
    \item
        The polynomial through the points
        \begin{align}
            (x,u(x)) \qquad (x + h, u(x+h)) \qquad \qty(x - \frac{h}{2}, u\qty(x - \frac{h}{2}))
        \end{align}
        satisfies
        \begin{align}
            u(x) &= A + Bx + Cx^2 \\
            u(x+h) &= A + B(x+h) + C(x+h)^2 \\
            u\qty(x - \frac{h}{2}) &= A + B\qty(x - \frac{h}{2}) + C\qty(x - \frac{h}{2})^2
        \end{align}
        This is a simple linear algebra problem that can be solved symbolically using a symbolic solver like Maple or Python.  We get:
        \begin{align}
            A &= \frac{1}{3 h^{2}} \left(3 h^{2} u(x) - 3 h u(x) x - h u(x + h) x + 4 h u\qty(x - \frac{h}{2}) x - 6 u(x) x^{2} + 2 u(x + h) x^{2} + 4 u\qty(x - \frac{h}{2}) x^{2}\right) \\
            B &= \frac{1}{3 h^{2}} \left(3 h u(x) + h u(x + h) - 4 h u\qty(x - \frac{h}{2}) + 12 u(x) x - 4 u(x + h) x - 8 u\qty(x - \frac{h}{2}) x\right)\\
            C &= \frac{1}{3 h^{2}} \left(- 6 u(x) + 2 u(x + h) + 4 u\qty(x - \frac{h}{2})\right)
        \end{align}
\end{enumerate}

\end{document}









