\documentclass{article}


\usepackage[margin=0.6in]{geometry}
\usepackage{amssymb, amsmath, amsfonts}
\usepackage{tabularx}
\usepackage{arydshln}
\usepackage{mathtools}
\usepackage{cancel}
\usepackage{physics}
\usepackage{enumerate}
\usepackage{enumitem}
\usepackage{nth}
\usepackage{array}
\usepackage{tikz}
\usepackage{pgfplots}
\newcommand{\enth}{$n$th}
\newcommand{\Rl}{\mathbb{R}}
\newcommand{\sgn}{\text{sgn}}
\newcommand{\ran}{\text{ran}}
\newcommand{\E}{\varepsilon}
\newcommand{\f}[3]{#1\ :\ #2 \rightarrow #3}

\newcommand{\tridsym}[3]{
    \qty(\begin{array}{ccccc}
                    #1 & #2 & & & \\
                    #3 & #1 & #2 & & \\
                    & \ddots & \ddots & \ddots &  \\
                    & & #3 & #1 & #2 \\
                    & & & #3 & #1
                \end{array})
}


\DeclareMathOperator*{\esssup}{\text{ess~sup}}

\title{MAT 228A Notes}
\author{Sam Fleischer}
\date{October 27, 2016}

\begin{document}
    \maketitle

    \section{Last Time - Iterative Solvers}
        \begin{itemize}
            \item Jacobi Iterations
            \begin{align*}
                u_{i,j}^{k+1} = \frac{1}{4}\qty(u_{i-1,j}^k + u_{i,j-1}^k + u_{i+1,j}^k + u_{i,j+1}^k - h^2f_{i,j})
            \end{align*}
            \item GS Lex
            \begin{align*}
                u_{i,j}^{k+1} = \frac{1}{4}\qty(u_{i-1,j}^{k+1} + u_{i,j-1}^{k+1} + u_{i+1,j}^k + u_{i,j+1}^k - h^2f_{i,j})
            \end{align*}
        \end{itemize}

    \section{Analysis of Jacobi}
        \begin{align*}
            u^{k+1} = u^k + Br^k \qquad \text{where} \qquad B \approx A^{-1}
        \end{align*}
        Now lets take $B = D^{-1} = -\frac{h^2}{4}I$ where $I$ is the identity.
        \begin{align*}
            u^{k+1} = u^k - \frac{h^2}{4}\qty(f - Au^k) = \qty(I + \frac{h^2}{4}A)u^k - \frac{h^2}{4}
        \end{align*}
        Both Jacobi and GS $u^{k+1} = Tu^k + c$ converges iff $\rho(T) < 1$.  Set $T_J = I + \frac{h^2}{4}A$.  So the spectrum of $T_J$ is the rescaled and shifted spectrum of $A$.  If $\lambda$ is an eigenvalue of $A$, then $1 + \frac{h^2}{4}\lambda$ is an eigenvalue of $T_J$.  The eigenfunctions of $A$ are
        \begin{align*}
            u_{i,j}^{\ell m} = \sin(\ell \pi x_i)\sin(m \pi y_j)
        \end{align*}
        with eigenvalues
        \begin{align*}
            \lambda^{\ell m} = \frac{2}{h^2}\qty(\cos(\ell \pi h) + \cos(m \pi h) - 2)
        \end{align*}
        Thus the eigenvalues for $T_J$, $\mu^{\ell m}$ are
        \begin{align*}
            \mu^{\ell m} = \frac{1}{2}\qty(\cos(\ell \pi h) + \cos(m \pi h))
        \end{align*}
        Since $h = \frac{1}{n+1}$ and $\ell,m = 1,2,\dots,n$, we have $\abs{\mu^{\ell m}} < 1$.  So the spectral radius is $\cos(\pi h)$.  However,
        \begin{align*}
            \cos(\pi h) = 1 - \frac{\pi^2h^2}{2} + \order{h^4}
        \end{align*}
        As $h \rightarrow 0$, $\rho(T_J) \rightarrow 0$, i.e.~convergence slows down.

    \section{GS Lex}
        When $f = 0$,
        \begin{align*}
            u_{i,j}^{k+1} = \frac{1}{4}\qty(u_{i-1,j}^{k+1} + u_{i,j-1}^{k+1} + u_{i+1,j}^k + u_{i,j+1}^k)
        \end{align*}
        So $T_{GS} = \qty(D - L)^{-1}U$.  Let $v^k$ be an eigenvector of $T_{GS}$ with eigenvalue $\lambda$.  So $v^{k+1} = \lambda v^k$.
        \begin{align*}
            \lambda v_{i,j} = \frac{1}{4}\qty(\lambda v_{i-1,j} + \lambda v_{i,j-1} + v_{i+1,j} + v_{i,j+1})
        \end{align*}
        Now change variables: let $v_{i,j} = \lambda^{\frac{i+j}{2}}u_{i,j}$.  Plugging this in above, we get
        \begin{align*}
            \lambda^{\frac{i+j}{2}+1}u_{i,j} &= \frac{1}{4}\qty(\lambda^{\frac{i+j-1}{2}+1}u_{i-1,j} + \lambda^{\frac{i+j-1}{2}+1}u_{i,j-1} + \lambda^{\frac{i+j+1}{2}}u_{i+1,j} + \lambda^{\frac{i+j+1}{2}}u_{i,j+1}) \\
            \implies \lambda^\frac{1}{2}u_{i,j} &= \frac{1}{4}\qty(u_{i-1,j} + u_{i,j-1} + u_{i+1,j} + u_{i,j+1})
        \end{align*}
        Thus $\lambda^\frac{1}{2} = \mu$ where $\mu$ is an eigenvalue of the Jacobi iteration.  So for $\mu$ an eigenvalue of Jacobi, we have $\mu^2$ is an eigenvalue of GS Lex. Thus $\rho(T_S) = \cos(\pi h) \implies \rho(T_{GS}) = \cos^2(\pi h) = (1 - \frac{\pi^2 h^2}{2} + \order{h^4})^2 = 1 - \pi^2h^2 + \order{h^4}$.

    \section{Error Analysis}
        Consider an iteration matrix $T$ with a complete set of eigenvectors (diagonalizable) with eigenvalues $\abs{\lambda_1} > \abs{\lambda_2} \geq \abs{\lambda_3} \geq \dots \geq \abs{\lambda_N}$.  Then recall $e^{k+1} = T e^k$ where $e^k$ is alebraic error.  Then $e^k = T^k e^0$.  We can express $e^0$ as a linear combination of the eigenvectors,
        \begin{align*}
            e^0 = \sum_{j=1}^N \alpha_j v_j
        \end{align*}
        where $v_j$ is the $j$th eigenvector of $T$.  Thus $e^k = \sum_{j-1}^N \lambda_j^k\alpha_j v_j$.
        \begin{align*}
            e^k = \lambda_1^k\alpha_1v_1 + \sum_{j=2}^N\lambda_j^k\alpha_jv_j = \lambda_1^k\qty(\alpha_1v_1 + \sum_{j=2}^N\qty(\frac{\lambda_j}{\lambda_1})^k\alpha_jv_j)
        \end{align*}
        So for $k$ large, we get $e^k \approx \lambda_1^k a_1 v_1$.  Lots of iterations?  For $k$ large,
        \begin{align*}
            \frac{\norm{e^{k+1}}}{\norm{e^k}} \approx \lambda_1 = \rho
        \end{align*}

        How many iterations to reduce the error by $\E$?  $\rho^k = \E$.  So $k = \dfrac{\log \E}{\log\rho}$.  How many iterations per digit of accuracy?  So for $\E = 0.1$?  $\dfrac{\log(10^{-1})}{\log\rho} = \frac{1}{10\log\rho}$.  So
        \begin{align*}
            \begin{array}{c|c|c}
                \text{grid} & \text{Jacobi} & \text{GS Lex} \\\hline
                32\times 32 & 507 & 254 \\
                64\times 64 & 1971 & 985 \\
                128\times 128 & 7764 & 3882 \\
                256\times 256 & 30818 & 15409
            \end{array}
        \end{align*}

        Spectral radius of Jacobi
        \begin{align*}
            \ln\rho_J \approx \ln(1 - \frac{\pi^2 h^2}{2}) = -\frac{\pi^2h^2}{2} + \text{higher order terms}
        \end{align*}
        Spectral radius of GS Lex
        \begin{align}
            \ln\rho_{GS} \approx -\pi^2h^2 + \text{higher order terms}
        \end{align}

        Work scaling to reduce the error by a factor of $Ch^2$?  So $\rho^k = C h^2$.  Thus $k = \dfrac{\ln (Ch^2)}{\ln \rho} \approx \dfrac{\ln C + 2\ln}{-\pi^2 h^2} \approx B \ln (h) h^{-2}$.

        Asymptotically, $h \sim \dfrac{1}{n}$.  Thus $k = \order{n^2\ln(n^2)} = \order{N\ln N}$.  This is the work for iteration.  The total work to solve is $\order{N^2\ln N}$.  This is like, or a little slower than, a banded factorization.  This is slow, but uses very little memory.  One does not just use these in research.  But they are part of better methods.

\end{document}



















