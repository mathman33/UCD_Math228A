\documentclass{article}


\usepackage[margin=0.6in]{geometry}
\usepackage{amssymb, amsmath, amsfonts}
\usepackage{mathtools}
\usepackage{cancel}
\usepackage{physics}
\usepackage{enumerate}
\usepackage{array}
\usepackage{tikz}
\usepackage{pgfplots}
\newcommand{\Rl}{\mathbb{R}}
\newcommand{\sgn}{\text{sgn}}
\newcommand{\E}{\varepsilon}
\newcommand{\f}[3]{#1\ :\ #2 \rightarrow #3}

\DeclareMathOperator*{\esssup}{\text{ess~sup}}

\title{MAT 228A Notes}
\author{Sam Fleischer}
\date{October 6, 2016}

\begin{document}
    \maketitle

    \section{1-Dimensional Poisson Equation}
        \subsection{Recall}
            \begin{align}
                u_{xx} = f \qquad u(0) = \alpha \qquad u(1) = \beta
            \end{align}
            Uniform spacing between (and including) $0$ and $1$ with $N+2$ points and $h = \frac{1}{N+1}$.  This yielded the finite-difference approximation
            \begin{align}
                A\underline{u}^h = \underline{b}
            \end{align}
            How big is the error?
            \begin{align}
                \underline{e}^h = \underline{u}^h - \underline{u}_\text{sol}, \qquad \text{where} \qquad \qty(u_\text{sol})_j = u(x_j)
            \end{align}
            We hope that $\norm{\underline{e}^h} = \order{h^2}$.  So the general question is ``How do errors in our operators relate to errors in the discrete solution?''
        \subsection{How is the size of this error related to discretization error?}
            In general, $\underline{u}_\text{sol}$ is not a solution to $A\underline{u}^h = \underline{b}$.  We define the local trunctation error $\underline{\tau}^h$
            \begin{align}
                \underline{\tau}^h = A\underline{u}_\text{sol} - \underline{b}
            \end{align}
            So for $j = 2, \dots, N-1$,
            \begin{align}
                \tau_j^h &= \frac{1}{h^2}\qty[u(x_{j-1}) - 2u(x_j) + u(x_{j+1})] - f(x_j) \\
                &= \frac{1}{h^2}\qty[u(x_j - h) - 2u(x_j) + u(x_j+h)] - f(x_j) \\
                &= \underline{u_xx(x_j)} + \frac{h^2}{12}u^{(4)}(x_j) + \text{higher order terms} - \underline{f(x_j)} \qquad \qquad \text{(using Taylor expansions)}
            \end{align}
            The underlined terms balance, so
            \begin{align}
                \tau_j^h = \frac{h^2}{12}u^{(4)}(x_j) + \order{h^4}
            \end{align}
            Now write
            \begin{align}
                A\underline{u}_\text{sol} &= \underline{b} + \underline{\tau}^h \\
                A\underline{u}_h = \underline{b}
            \end{align}
            Subtracting these equations, and usin the fact that $\underline{e}^h = \underline{u}^h - \underline{u}_\text{sol}$, gives
            \begin{align}
                A(\underline{u}^h - \underline{u}_\text{sol}) &= -\underline{\tau}^h \\
                A\underline{e}^h &= -\underline{\tau}^h \\
                \underline{e}^h &= -A^{-1}\underline{\tau}^h
            \end{align}
        \subsection{Consistency vs. Convergence}
            A numerical scheme is ``consistent'' if the trunctation error goes to $0$, i.e.~if $\underline{\tau}^h \rightarrow 0$ as $h \rightarrow 0$.  A numerical scheme is ``convergence'' if $\underline{e}^h \rightarrow 0$ as $h \rightarrow 0$.  {\color{blue}A scheme can be consistent and not convergent if $\norm{A^{-1}} = \infty$.}

            For linear schemes applied to linear PDEs we have the Lax-Equivalence Theorem:
            \begin{itemize}
                \item If a scheme is consistent and stable, then it is convergent.
            \end{itemize}
        \subsection{Stability}
            ``Stable'' in our case means that
            \begin{align}
                \norm{(A^h)^{-1}} \leq C \qquad \text{for all } h \leq h_0 \text{ where $C$ is independent of $h$}
            \end{align}
            \begin{align}
                \norm{\underline{e}^h} = \norm{A^{-1}\underline{\tau}^h} \leq \norm{A^{-1}}\norm{\underline{\tau}^h} \leq C\norm{\underline{\tau}^h}b
            \end{align}
            \subsubsection{Stability in the 2-norm}
                Remember $A$ is symmetric, and that $\norm{A}_2 = \rho(A) = \max\abs{\lambda_i}$.  $\sqrt{\rho\qty(AA^*)} = \sqrt{\rho\qty(A^2)} = \rho\qty(\sqrt{A^2}) = \rho(A)$.  Then
                \begin{align}
                    \norm{A^{-1}}_2 = \rho\qty(A^{-1}) = \max\abs{\frac{1}{\lambda_j}} = \frac{1}{\min\abs{\lambda_j}}
                \end{align}
            \subsubsection{Explicitly Calculating the Eigenvalues of $A$}
                Recall $u^k(x) = \sin(k\pi x)$ is an eigenfunction of $\frac{\dd^2}{\dd x^2}$ on functions zero at $0$ and $1$.  It turns out the eigenvectors of $A$ are these eigenfunctions evaluated on the grid points.  So we claim
                \begin{align}
                    u_j^k = \sin(k\pi x_i), \qquad 1, \dots, N
                \end{align}
                are eigenvectors of $A$.  Let's explicitly show this.
                \begin{align}
                    \frac{1}{h^2}\qty[\sin(k\pi x_{j-1}) - 2\sin(k\pi x_j) + \sin(k \pi x_{j+1})] &= \frac{1}{h^2}\qty[\sin(k\pi(x_j - h)) - 2\sin(k\pi x_j) + \sin(k\pi x_j+h)] \\
                    &= \frac{1}{h^2}\qty[\sin(k\pi x_j)\cos(k\pi h) - \sin(k\pi h) \cos(k\pi x_j)] \\
                    &\qquad - \frac{1}{h^2}\qty[-2\sin(k\pi x_j)] \\
                    &\qquad + \frac{1}{h}^2\qty[\sin(k\pi x_j)\cos(k\pi h) + \sin(k\pi h)\cos(k\pi x_j)] \\
                    &= \frac{1}{h^s}\qty(2\cos(k\pi h) - 2)\sin(k\pi x_j)
                \end{align}
                Thus we have eigenvalues
                \begin{align}
                    \lambda_k &= \frac{2}{h^2}\qty(\cos(k\pi h) - 1), \qquad k = 1, \dots, N \\
                    &= -\frac{4}{h^2}\sin^2\qty(\frac{k\pi h}{2})
                \end{align}
                As $k$ goes from $1$ to $N$, $k\pi h$ goees from $\pi h$ to $N\pi h = \frac{N\pi}{N+1}$.  So, the smallest magnitude eigenvalue is
                \begin{align}
                    \lambda_1 &= \frac{2}{h^2}\qty(\cos(\pi h) - 1) \\
                    &= \frac{2}{h^2}\qty(1 - \frac{1}{2}\pi^2 h^2 + \order{h^4} - 1) \\
                    &= -\pi^2 + \order{h^2}
                \end{align}
                The following is in red since I was falling asleep and cannot vouch for accuracy...
                
                {\color{red}So we have control of this inverse since
                                \begin{align}
                                    \norm{A^{-1}}_2 = \frac{1}{\pi^2} + \order{h^2}
                                \end{align}
                                and thus
                                \begin{align}
                                    \underline{e}^h &= -\norm{A^{-1}}_2\norm{\underline{\tau}^h}_2 \\
                                    &= \qty(\frac{1}{\pi^2} + \order{h^2})(\order{h@2})
                                \end{align}
                        \subsection{Eigenvalues of the Continuous space operator are}
                            \begin{align}
                                \lambda_k^c = -k^2\pi^2
                            \end{align}}

\end{document}



















